documentclass[a4paper]{article}

usepackage[T1]{fontenc}
usepackage[utf8]{inputenc}

usepackage[polish]{babel}

usepackage{amsmath, amsfonts}
usepackage{indentfirst}
usepackage[nofoot,hdivide={2cm,,2cm},vdivide={2cm,,2cm}]{geometry}
frenchspacing
pagestyle{empty}
setlengthparindent{0pt}

begin{document}

Large {textbf{Marcin Sarnecki 323034}}
large
vspace{0.5cm}

Large{Lista 5, zadanie 5}
large
vspace{0.5cm}

Wygrana jest zawsze możliwa dla n postaci $3a$ oraz $3a+1$, gdzie $ain N_+$ newline
O dostępnych ruchach możemy myśleć jak o wektorach nad ciałem $Z_2$
vspace{1cm}

begin{bmatrix}
1 & 1 &   &   &   &   &  
1 & 1 & 1 &   &   &   &  
  & 1 & 1 & 1 &   &   &  
  &   & 1 & 1 & 1 &   &  
  &   &   & 1 & 1 & 1 &  
  &   &   &   & 1 & 1 & 1
  &   &   &   &   & 1 & 1
end{bmatrix}
hspace{0.7cm}xrightarrow{}hspace{0.7cm}
begin{bmatrix}
1 & 1 &   &   &   &   &  
  & 1 & 0 & 1 &   &   &  
  &   & 1 &   &   &   &  
  &   &   & 1 & 1 &   &  
  &   &   & 1 & 1 & 1 &  
  &   &   &   & 1 & 1 & 1
  &   &   &   &   & 1 & 1
end{bmatrix}

vspace{0.4cm}
hspace{3cm}(2 - 1, 2 + 3, 3 - 2, 4 - 3)

vspace{0.2cm}

Operacjami elementarnymi doprowadzam pierwsze 3 wiersze do postaci schodkowej, reszta wierszy wygląda tak samo jak początek macierzy, więc mogę wykonywać te same operacje elementarne dla każdych kolejnych trzech wierszy

vspace{0.6cm}

W przypadkach $n=3a$ oraz $n=3a+1$ na końcu otrzymamy postać schodkową

vspace{0.1cm}

hspace{3.2cm} $n=3a$ hspace{7.8cm} $n=3a+1$

vspace{0.1cm}

begin{bmatrix}
1 &   &   &   &   &  
  & 1 &   &   &   &  
  &   & 1 &   &   &  
  &   &   & 1 & 1 &  
  &   &   & 1 & 1 & 1
  &   &   &   & 1 & 1
end{bmatrix}
hspace{0.1cm}xrightarrow{}hspace{0.1cm}
begin{bmatrix}
1 &   &   &   &   &  
  & 1 &   &   &   &  
  &   & 1 &   &   &  
  &   &   & 1 &   &  
  &   &   &   & 1 &  
  &   &   &   &   & 1
end{bmatrix}
hspace{1cm}
begin{bmatrix}
1 &   &   &   &   &   &  
  & 1 &   &   &   &   &  
  &   & 1 &   &   &   &  
  &   &   & 1 & 1 &   &  
  &   &   & 1 & 1 & 1 &  
  &   &   &   & 1 & 1 & 1
  &   &   &   &   & 1 & 1
end{bmatrix}
hspace{0.1cm}xrightarrow{}hspace{0.1cm}
begin{bmatrix}
1 &   &   &   &   &   &  
  & 1 &   &   &   &   &  
  &   & 1 &   &   &   &  
  &   &   & 1 &   &   &  
  &   &   &   & 1 &   &  
  &   &   &   &   & 1 &  
  &   &   &   &   &   & 1
end{bmatrix}

vspace{0.5cm}

Zatem w przypadkach $n=3a$ oraz $n=3a+1$ za pomocą dostępnych ruchów możemy uzyskać dowolną kombinację 0 i 1

vspace{0.5cm}

Przypadek $n=3a+2$ jest inny
vspace{0.3cm}


begin{bmatrix}
1 &   &   &   &   &   &   & 
  & 1 &   &   &   &   &   & 
  &   & 1 &   &   &   &   & 
  &   &   & 1 & 1 &   &   & 
  &   &   & 1 & 1 & 1 &   & 
  &   &   &   & 1 & 1 & 1 & 
  &   &   &   &   & 1 & 1 & 1
  &   &   &   &   &   & 1 & 1
end{bmatrix}
hspace{0.1cm}xrightarrow{}hspace{0.1cm}
begin{bmatrix}
1 &   &   &   &   &   &   & 
  & 1 &   &   &   &   &   & 
  &   & 1 &   &   &   &   & 
  &   &   & 1 &   &   &   & 
  &   &   &   & 1 &   &   & 
  &   &   &   &   & 1 &   & 
  &   &   &   &   & 1 & 1 & 1
  &   &   &   &   &   & 1 & 1
end{bmatrix}
hspace{0.1cm}xrightarrow{}hspace{0.1cm}
begin{bmatrix}
1 &   &   &   &   &   &   & 
  & 1 &   &   &   &   &   & 
  &   & 1 &   &   &   &   & 
  &   &   & 1 &   &   &   & 
  &   &   &   & 1 &   &   & 
  &   &   &   &   & 1 &   & 
  &   &   &   &   & 0 & 0 & 0 
  &   &   &   &   &   & 1 & 1
end{bmatrix}

vspace{0.5cm}

Pojawił się wiersz zerowy, nie uzyskamy wszystkich kombinacji 0 i 1. Zatem w niektórych przypadkach $n=3a+1$ wygrana nie jest mozliwa, np. $00001$

vspace{0.5cm}

} {

Prosty algorytm zachłanny (dla przypadków $3a$ oraz $3a+1$)
begin{itemize}
  item Po kolei wykonujemy ruchy tak, aby wszystko z lewej strony było zapalone aż do ostatnich 3 pól
  item Pole underline{$n-2$} możemy zmieniać niezależnie od innych pól poprzez operacje na polach  underline{$n$} oraz  underline{$n-1$}
  item Jedno z pól {underline{$n-1$}, underline{$n$}} możemy zmienić niezależnie od innych pól w następujący sposób
  begin{itemize}
  item ustawiamy pole underline{$3$} operacjami na pozycjach underline{$1$} i underline{$2$}
  item zmianę na polu 3 przesuwamy dalej co 3 pola za pomocą 2 ruchów 
  
  vspace{0.5cm}
  
  begin{matrix}
   &   &   & 1 & 1 & 1 &   &   
   &   & 1 & 1 & 1 &   &   &   
 0 & 0 & 1 & 0 & 0 & 0 & 0 & 0 
(1)&(2)&(3)&(4)&(5)&(6)&(7)&(8)
end{matrix}
hspace{0.1cm}xrightarrow{}hspace{0.1cm}
begin{matrix}
   &   &   &   &   &   &   &   
   &   &   &   &   &   &   &   
 0 & 0 & 0 & 0 & 0 & 1 & 0 & 0 
(1)&(2)&(3)&(4)&(5)&(6)&(7)&(8)
  end{matrix}
  end{itemize}
  vspace{0.2cm}
  
  Na końcu trafimy na pole underline{$n-1$} (w przypadku $3a+1$) lub na pole underline{$n$} (w przypadku $3a$)
  Zatem jeśli możemy niezależnie od innych pól zmienić dwa pola spośród ostatnich 3 pól, dokładając do tego operację na polu underline{$n-1$} uzyskujemy możliwość dowolnego ustawienia 3 ostatnich pól, zatem jesteśmy w stanie zapalić wszystkie pola niezależnie od początkowego ustawienia
  
 
end{itemize}
end{document}